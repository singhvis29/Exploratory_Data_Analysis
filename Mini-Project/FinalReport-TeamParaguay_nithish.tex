\documentclass[]{article}
\usepackage{lmodern}
\usepackage{amssymb,amsmath}
\usepackage{ifxetex,ifluatex}
\usepackage{fixltx2e} % provides \textsubscript
\ifnum 0\ifxetex 1\fi\ifluatex 1\fi=0 % if pdftex
  \usepackage[T1]{fontenc}
  \usepackage[utf8]{inputenc}
\else % if luatex or xelatex
  \ifxetex
    \usepackage{mathspec}
  \else
    \usepackage{fontspec}
  \fi
  \defaultfontfeatures{Ligatures=TeX,Scale=MatchLowercase}
\fi
% use upquote if available, for straight quotes in verbatim environments
\IfFileExists{upquote.sty}{\usepackage{upquote}}{}
% use microtype if available
\IfFileExists{microtype.sty}{%
\usepackage{microtype}
\UseMicrotypeSet[protrusion]{basicmath} % disable protrusion for tt fonts
}{}
\usepackage[margin=1in]{geometry}
\usepackage{hyperref}
\hypersetup{unicode=true,
            pdftitle={Mini Project},
            pdfauthor={Srinithish Kandagadla and Vishal Singh},
            pdfborder={0 0 0},
            breaklinks=true}
\urlstyle{same}  % don't use monospace font for urls
\usepackage{graphicx,grffile}
\makeatletter
\def\maxwidth{\ifdim\Gin@nat@width>\linewidth\linewidth\else\Gin@nat@width\fi}
\def\maxheight{\ifdim\Gin@nat@height>\textheight\textheight\else\Gin@nat@height\fi}
\makeatother
% Scale images if necessary, so that they will not overflow the page
% margins by default, and it is still possible to overwrite the defaults
% using explicit options in \includegraphics[width, height, ...]{}
\setkeys{Gin}{width=\maxwidth,height=\maxheight,keepaspectratio}
\IfFileExists{parskip.sty}{%
\usepackage{parskip}
}{% else
\setlength{\parindent}{0pt}
\setlength{\parskip}{6pt plus 2pt minus 1pt}
}
\setlength{\emergencystretch}{3em}  % prevent overfull lines
\providecommand{\tightlist}{%
  \setlength{\itemsep}{0pt}\setlength{\parskip}{0pt}}
\setcounter{secnumdepth}{0}
% Redefines (sub)paragraphs to behave more like sections
\ifx\paragraph\undefined\else
\let\oldparagraph\paragraph
\renewcommand{\paragraph}[1]{\oldparagraph{#1}\mbox{}}
\fi
\ifx\subparagraph\undefined\else
\let\oldsubparagraph\subparagraph
\renewcommand{\subparagraph}[1]{\oldsubparagraph{#1}\mbox{}}
\fi

%%% Use protect on footnotes to avoid problems with footnotes in titles
\let\rmarkdownfootnote\footnote%
\def\footnote{\protect\rmarkdownfootnote}

%%% Change title format to be more compact
\usepackage{titling}

% Create subtitle command for use in maketitle
\providecommand{\subtitle}[1]{
  \posttitle{
    \begin{center}\large#1\end{center}
    }
}

\setlength{\droptitle}{-2em}

  \title{Mini Project}
    \pretitle{\vspace{\droptitle}\centering\huge}
  \posttitle{\par}
    \author{Srinithish Kandagadla and Vishal Singh}
    \preauthor{\centering\large\emph}
  \postauthor{\par}
      \predate{\centering\large\emph}
  \postdate{\par}
    \date{October 3, 2019}

\usepackage{booktabs}
\usepackage{longtable}
\usepackage{array}
\usepackage{multirow}
\usepackage{wrapfig}
\usepackage{float}
\usepackage{colortbl}
\usepackage{pdflscape}
\usepackage{tabu}
\usepackage{threeparttable}
\usepackage{threeparttablex}
\usepackage[normalem]{ulem}
\usepackage{makecell}
\usepackage{xcolor}

\begin{document}
\maketitle

\hypertarget{abstract}{%
\subsubsection{Abstract}\label{abstract}}

In this project we study the change in two of the main indicators of
development level of a country i.e.~Life Expectancy and Gross Domestic
Product (GDP) per capital for years after the Second World War. The two
metrics are defined as follows- 1. \textbf{Life Expectancy} - It refers
to the number of years a person is expected to live. Life expectancy is
based on an estimate of the average age that members of a particular
population group will be when they die. 2. \textbf{Gross Domestic
Product} - It is the total monetary or market value of all the finished
goods and services produced within a country's borders in a specific
time period. \textbf{GDP per capita} is a measure of a country's
economic output that accounts for its number of people. We measure these
metrics across years and each other to see the trend and how they effect
each other. Further, we study their effects across various continent to
measure how have the different countries in the continents have
performed and the overall performance of the continents. This helps us
measure if some parts of the world are performing better than the others
and if all the countries have performed similarly in a continent or if
some countries are better than others. In general, we find that both the
GDP per capita and Life Expectancy has increased after World War 2, this
highlights a high correlation between the two metrics. We zoom in on all
the continents and countries to provide a more granular view on these
metrics.

\hypertarget{data}{%
\subsubsection{Data}\label{data}}

We obtain the data from the gapminder website, we download the tables
which have the longitudinal data for the metrics - Life Expectancy,
Gross Domestic Product per capita, and Population. The table for Life
Expectancy contains data for different countries for years 1800-2018.
The value for each year appears as a seperate column. We used the data
for post WW2 years for our project, this mean only years after 1945 were
used. The table for GDP/per capita contains data for different countries
for years 1800-2040. The years 1945-2018 are filtered for our analysis.
This table also contains data for different years as different columns.
This table contains inflation-adjusted value and can therefore be
compared with each other directly. The table for population contains
country population for years 1800-2100 in different columns. We filter
for years 1945-2100 We also use the country-continent mapping form the
gapminder data in the gapminder library. This table is joined with the
various tables to aggregate the information at the continent level

\hypertarget{problem-statement}{%
\subsubsection{Problem Statement}\label{problem-statement}}

We broadly look at 3 problem statements to measure the metrics in
question and explain the change w.r.t. time and each other 1. How were
the GDP and Life Expectancy related for the year 2018 ? 2. How did the
Average Life Expectancy change w.r.t. to time for different continents ?
3. How has the relationship between GDP per capita and Life Expectancy
changed over years? We further drill down on these problem statements to
look at the metrics at a more granular level for different countries.

\hypertarget{question-1}{%
\subsection{Question 1}\label{question-1}}

\begin{center}\includegraphics{FinalReport-TeamParaguay_nithish_files/figure-latex/unnamed-chunk-2-1} \end{center}

In Figure 1 we look at Life Expectancy vs log of GDP per capital for the
year 2018 for all the countries. We try to fit a \textbf{Linear Model}
and \textbf{Loess Model} on this data. We can see that if we take the
log of value for GDP in 2019 and plot Life Expectancy against it, both
linear model and loess model explain the data well. This can be seen
from the fact that the R-squared value for the linear model is
\textasciitilde{}0.74.

\begin{table}[t]

\caption{\label{tab:unnamed-chunk-3}Table 1 - Continent statistics for year 2018}
\centering
\begin{tabular}{l|r|r|r|r|r}
\hline
Continent & Avg Life Expectancy & Avg GDP & Life Expectancy Variance & GDP variance & Correlation (LE and GDP)\\
\hline
Africa & 64.97451 & 5100.392 & 31.343137 & 31233179 & 0.5491858\\
\hline
Americas & 76.70833 & 16757.083 & 13.581667 & 150263091 & 0.5036482\\
\hline
Asia & 74.45185 & 20630.000 & 33.856439 & 447932362 & 0.6773218\\
\hline
Europe & 80.11000 & 34923.333 & 6.344379 & 206712195 & 0.7378562\\
\hline
Oceania & 82.35000 & 41100.000 & 0.605000 & 44180000 & 1.0000000\\
\hline
\end{tabular}
\end{table}

\textbf{Overall Average Life Expectancy - 72.63}

\textbf{Overall Average GDP - 13307.63}

From Figure 1 we can see that for the year 2018 the relationship between
GDP and life expectancy can be explained using a linear regression line
for all the continents. From Table 1 we can also observe good
correlation (\textgreater{} 0.50) between the two variables for all the
continents. We observe the following pattern for the different
continents-

\begin{enumerate}
\def\labelenumi{\arabic{enumi}.}
\item
  In Africa we can see GDP \textless{} 15000 and life expectancy
  \textless{} 75 for most of the countries. The mean Life Expectancy and
  GDP is lowest amongst all the continents, these metrics are also lower
  than the overall averages. Only a few countries are above the overall
  average for GDP - Botswana, Gabon, Libya and Mauritius; and only 3
  countries - Algeria, Egypt and Mauritius are above in term of average
  Life Expectancy than overall average. From table 1, we can observe
  least variance in the GDP for countries in Africa continent. The slope
  for Africa is the highest which means that effect of increase in Life
  Expectancy is the highest with increase in GDP. 
\item
  For America, we can see a couple of outliers - US, Canada for GDP
  which have values for GDP \textgreater{} 40k. The mean life expectancy
  if well above overall average, the GDP per capita is close to the
  world average. We see can a lot of countries in the range of 78-80+
  for this continent which states that countries not just US and Canada,
  but others as well have a good status of health, this is also
  supported by the fact that variance for Life Expectancy is quite low.
  We can see a strong correlation (0.50) between the variable for this
  continent as well. 
\item
  We can see a lot of countries with a very high GDP in Asia, these
  values might have been reported wrongly as they don't match with the
  data online. We can see a very high value of correlation between life
  expectancy and GDP for Asia (0.75), this can be evident from cases
  like Japan where life expectancy and GDP both are high and Afghanistan
  where both the metrics are low. 
\item
  We can see the highest mean values of life expectancy and GDP in the
  case of Europe and the value for correlation is also the highest for
  this continent indicating a strong relationship 
\end{enumerate}

The slope for the linear regression line between life expectancy and GDP
is the highest for Africa indicating that Life Expectancy increase the
most with increase in GDP in this continent. The pattern can be
described using linear model for all the continents. This is followed by
Asia where we can see countries with very high GDP and Life Expectancy
and also onces with very low values.

\begin{center}\includegraphics{FinalReport-TeamParaguay_nithish_files/figure-latex/unnamed-chunk-4-1} \end{center}

\hypertarget{question-2}{%
\subsection{Question 2}\label{question-2}}

\begin{enumerate}
\def\labelenumi{\arabic{enumi}.}
\setcounter{enumi}{1}
\tightlist
\item
  We can see from the Figure 3 that a linear model explains very well
  the increase in life expectancy for years after the second world war
  in 1945. This has been the case for all the continents, and we can see
  that all the continents have caught up in term of life expectancy
  except for Africa which has a slightly lower average life expectancy
  in the year 2018 (highest). We can see a great improvement in the
  average life expectancy in case of Asia (35 to 73) and Africa (35 to
  66) We can observe a slight dip for Europe in years after the war,
  this could be because Europe was at the center of the war and could
  have suffered from the aftermath of the war. We can also see a slight
  upwards bump in case of Asia in the 1960s, this could be due to
  improvement of economy in the middle east due to the rise in exports
  of oil. There is a downwards bump for Asia prior to this period, this
  could be due to incorrect data as the magnitude of the bump is big. A
  liner model is enough to explain the increase in the life expectancy
  over the years. Overall we can say that the world is doing better than
  what it was in 1945 in terms of life expectancy . As evident from the
  table, we can see there is a lot of variation in the case of Asia for
  both life expectancy and GDP this is due to countries having extreme
  values for GDP like Afghanistan and the middle eastern countries, from
  the plot we can see the maximum amount of deviation from the standard
  regression line in case of Asia.
\end{enumerate}

\begin{center}\includegraphics{FinalReport-TeamParaguay_nithish_files/figure-latex/unnamed-chunk-6-1} \end{center}

Since, the continents have more or less caught up in term of life
expectancy we look at the data for recent years to check if its due to
all the countires in the continent. In Figure 4, we have plotted the
boxplots for different continents for data after year 2014, we see that
that the interquartile range is the highest for Asia and Africa and
lowest for Americas and Europe, this shows that the disparity is highest
in continents of Asia and Africa and lowest in America and Europe. In
the case of Asia, the life expectancy of some countries is among the
highest e.g.- Japan, Singapore and lowest e.g.- Afghanistan

\begin{center}\includegraphics{FinalReport-TeamParaguay_nithish_files/figure-latex/unnamed-chunk-7-1} \end{center}

Next in Figure 5, we look at breakdown of slopes over different periods
after WW2 for different continents

\begin{itemize}
\tightlist
\item
  In case of Africa, we can see the greatest slope in the period of
  2003-2018, which means the life expectancy increased the most in this
  period
\item
  For America, slope for the Avg Life Expectancy vs Time graph has
  decreased over time, this means that the improvement in life
  expectancy is approaching stagnation
\item
  We saw the greatest increase in life expectancy for the case of Asia
  in the period of 1960 -1974, it has continued to show a high positive
  slope ever since
\item
  We see an increase in life expectancy over years for Europe as well
  *The number of data points for Ocenia are few but we can see a
  positive slope for all time periods
\end{itemize}

\begin{center}\includegraphics{FinalReport-TeamParaguay_nithish_files/figure-latex/unnamed-chunk-8-1} \end{center}

\hypertarget{question-3}{%
\subsection{Question 3}\label{question-3}}

In Figure 6, we look at Weighted Average Life Expectancy vs Weighted
Average GDP per capita across different continents for different years.
The weighted average GDP per capita is less for Asia as the countries
with largest populations (India, China) have low GDP. We can deduce the
following for different continents from the figure-

\begin{itemize}
\item
  In case of Africa, increase of Average GDP over the years has resulted
  in increase in average Life Expectancy, there has been a stark
  increase in the avg life expectancy over the years 
\item
  For America, there has been a steady improvement in Life Expectancy
  with increase in GDP over the years 
\item
  For Asia, there is a sharp increase average life expectancy with
  increase in GDP in years until 1980 after that the increase in life
  expectancy has slowed with increase in average GDP 
\item
  In case of Europe there has been a steady increase in Life Expectancy
  with increase in GDP over the years until the life expectancy has
  reached \textasciitilde{}80 years 
\item
  There has been a slow increase in average life expectancy for Oceania
  in years prior to 1970 with increase in GDP, the rate increased during
  years between 1970 - 1990 but has slowed since then 
\end{itemize}

Overall we can say that prior to 1980s, both average GDP and average
Life Expectancy increased rapidly for most countries, which meant that
GDP was Life Expectancy a lot during years. The increase in Life
Expectancy has slowed since then with more and more countries reaching a
life expectancy of 80

\begin{center}\includegraphics{FinalReport-TeamParaguay_nithish_files/figure-latex/unnamed-chunk-10-1} \end{center}

\hypertarget{references}{%
\subsubsection{References}\label{references}}

\begin{itemize}
\tightlist
\item
  \url{https://ourworldindata.org/life-expectancy-how-is-it-calculated-and-how-should-it-be-interpreted}
\item
  \url{https://www.investopedia.com/terms/g/gdp.asp}
\item
  \url{https://rmarkdown.rstudio.com/authoring_basics.html}
\item
  \url{https://www.gapminder.org/data/}
\item
  \url{https://rmarkdown.rstudio.com/lesson-7.html}
\end{itemize}

\hypertarget{appendix}{%
\subsection{Appendix}\label{appendix}}

\begin{center}\includegraphics{FinalReport-TeamParaguay_nithish_files/figure-latex/unnamed-chunk-11-1} \end{center}

\begin{center}\includegraphics{FinalReport-TeamParaguay_nithish_files/figure-latex/unnamed-chunk-12-1} \end{center}


\end{document}
